\documentclass[oneside,final,14pt]{extreport}
\usepackage[utf8]{inputenc}
\usepackage[T2A]{fontenc}
\usepackage[english,russian]{babel}
\usepackage{vmargin}
\setpapersize{A4}
\setmarginsrb{1.5cm}{1.5cm}{1.2cm}{0.8cm}{0pt}{0mm}{0pt}{12mm}
\usepackage{indentfirst}
\usepackage{hyperref}

%\usepackage{amsmath}
%\usepackage{mathtools}
%\usepackage{breqn}
%\usepackage{split}
%\usepackage{aligned}
%\usepackage{gathered}
%\usepackage{alignedat}

\usepackage{amsmath}
\usepackage{amssymb}
\usepackage{amsfonts}
\usepackage{amstext}
\usepackage{latexsym}
\usepackage{bm}
\usepackage{url}

\usepackage{breqn}

%\DeclareMathOperator{\Sp}{Sp-t:r}
%\def\nobr{} % это позволит работать и в случае, если загрузка rumathbr будет закомментирована
%\usepackage{rumathbr}

\usepackage[matrix,arrow]{xy} % Xy - после rumathbr

\sloppy



\begin{document}

\tableofcontents

\chapter{Проект с рабочим названием Sphere}

\section{От Миши}

На экране iPhone~/ iPad изображен многогранник, усеченный икосаэдр, на каждой грани изображена обложка от фильма~--- пока нужно пробовать напрямую взаимодействовать с netflix, в частности у них есть методы по получению похожих фильмов: \\
\url{http://api.netflix.com/catalog/titles/movies/60011649/similars}, \\
\url{http://api.netflix.com/catalog/titles/movies/60011649?expand=similars}.

Нужен алгоритм, который получив на вход один фильм, находит 6 или 5 (в зависимости от грани) максимально разных фильмов из похожих~--- и располагает их вокруг~--- так, чтобы рядом не оказывались похожие. Алгоритм продумать, чтобы его можно было описать несколькими предложениями, потом рассказать мне. Дальше сделать возможность вращения данной фигуры, и подкачки фильмов с Netflix. Ограничить вращение~--- чтобы нельзя было крутануть так что связь потеряется... Хотя можно сделать фичу~--- если тебе ничего не нравится~--- резко крутишь, и тебе новое кино в центр. В центре пока с фиксированного начинаем~--- дальше надо будет сделать строку поиска сверху. Сразу же при выборе технологии предусмотреть, чтобы можно было вообще текстовые филды делать.

\section{Алгоритм заполнения граней}
	
Информация о фильмах берётся из сервиса \url{http://api.rottentomatoes.com}, предоставляемого сайтом: \url{http://developer.rottentomatoes.com} (логин: ikoznov; пароль: 4a78poi2; ключ приложения: gfh7w8cm88rvptkndqq556x4). Для обращения к сервису используются GET-запросы, в частности методы: \\
\url{http://api.rottentomatoes.com/api/public/v1.0/movies/movie_id/similar.json} для получения списка похожих фильмов на заданный; \\
\url{http://api.rottentomatoes.com/api/public/v1.0/movies.json} для поиска фильма по названию и получения его идентификатора.

Сервис также предоставляет ссылки на изображения с постерами фильмов различного качества. Алгоритм, определяющий какие именно постеры будут изображены на гранях можно разделить на 3 этапа.

Первый этап: получение необходимой информации из сервиса и запись, её в граф, в котором каждая вершина представляет собой фильм, а каждое ребро~--- прямую связь между двумя фильмами. После построения графа, происходит обработка полученной информации, с целью отбора постеров для отображения на гранях усечённого икосаэдра.

Итак, цель второго этапа~--- определение того, насколько сильно связаны друг с другом фильмы, находящиеся в графе. Для этого вводится некоторая функция, значения которой лежат строго в интервале от 0 до 1, чем больше значение функции тем сильнее связаны фильмы. Функция не может равняться нулю, поскольку граф связный, т.~е. между любыми двумя фильмами существует связь, пусть и косвенная. И единице функция также не может равняться, поскольку это бы обозначало, что рассматривается один и тот же фильм. Для вычисления значения функции связи используется количество и длина всех путей от одной заданной вершины до другой, т.~е. учитываются все связи между двумя фильмами, как прямые, так и косвенные, а также и степень их косвенности. Условимся, что вес прямой связи выше веса всех косвенных, и что вес любой косвенной связи выше совокупного веса любого количества связей большей длины. Исходя из этих условий значение функции связи можно представить, как сумму:
\begin{align} \begin{split}
\left ( \frac{1}{3} + \frac{1}{6} + \frac{1}{12} + \frac{1}{24} + \ldots \right ) +
\left ( \frac{1}{9} + \frac{1}{18} + \frac{1}{36} + \frac{1}{72} + \ldots \right ) + \\ +
\left ( \frac{1}{27} + \frac{1}{54} + \frac{1}{108} + \frac{1}{216} + \ldots \right ) + \,\ldots
\end{split} \end{align}
Что в общем случае можно представить как бесконечную сумму и доказать, что она стремится к 1:
\begin{equation} \lim \sum_{i=1}^{\infty} \sum_{j=1}^{\infty} \frac{2}{ 3^{i} 2^{j} } = 1 \end{equation}

Каждое слагаемое этой суммы соответствует весу определённой связи. Первая часть суммы, заключённая в скобки равняется весу всех прямых связей, связей единичной длины, на практике она всегда только одна. Следующие подсуммы соответственно равняются весу всех косвенных связей определённой длины. В результате второго этапа алгоритма подсчитывается матрица, которая содержит значения силы связи между всеми фильмами в графе.

На последнем, третьем, этапе непосредственно отбираются постеры для отображения. Происходит это по следующему критерию: вокруг заданного фильма должны оказаться фильмы с наибольшей с ним связью, но в то же время связи между набором этих фильмов должны быть минимальны. Отбор происходит только для не заполненных граней, грани для которых постер определён, не меняются.

\end{document}
